\documentclass[12pt]{report}

\usepackage[utf8]{inputenc}
\usepackage[T1]{fontenc}
\usepackage[french]{babel}
\usepackage{graphicx}

\begin{document}
\title{Cahier des charges utilisateur}
\author{Yoll Mu Krow}

\maketitle

\begin{abstract}
	Ceci est le cahier des charges du projet LiSA créée le samedi 18 mai et développé par Yoll Mu Krow.
	Le but de ce projet, en plus de continuer à m'entrainer avec les différentes utilisations et créations web. Est de créer un logiciel (comprenant, Base de données, interface web et serveur) qui permettra à terme de gérer toute les oeuvres vu, lu, désiré par un utilisateur et de les classer.
\end{abstract}
\newpage
\tableofcontents
\newpage

\section{Le Projet LiSA}
\textbf{Principe}
Comme dit plus haut, ce projet à pour but de créer une interface web qui permettra à un utilisateur de gérer/stocker/commenté, les différentes oeuvres qu'il à vu, lu ou qu'il désire voire, ...
\textbf{Ces fonctionnalités (V1)}
nb: C'est liste n'est pas exhaustive et pourra être modifier au fur et a mesure que les idées viennent.
Définition Oeuvres : Les oeuvres cités sont pour l'instant, des livres, animés, mangas, films, musiques.

\begin{itemize}
	\item Créer un compte
	\item Se connecter au compte
	\item Ajouter des œuvres 
	\item Modifier des œuvres
	\item Supprimer des œuvres
	\item Commenter une oeuvre
	\item Noter une oeuvre
	\item Pouvoir cliquer sur une oeuvre et être rediriger sur la page correspondante
	\item Ordonner les oeuvres
	\item Classer les oeuvres selon différent critères (genre/type/état/...)
\end{itemize}
Une compatibilités pour smartphone peut être envisagé.

\section{Les technologies}
\textbf{Organisation du logiciel}
Ce logiciel pourra ce trouver sur plusieurs formes:
\begin{itemize}
	\item Modèle Vue Contrôleur 
	\item Site web classique + serveur
	\item Logiciel + Page web + Base de données
\end{itemize}

\textbf{Les Frameworks}
Les frameworks pouvant être implémenté sont les suivants: 
\begin{itemize}
	\item JEE
	\item Django
	\item ASP.NET MVC
	\item ASP.NET
	\item Symphony
	\item Node.js
\end{itemize}

\textbf{Les langages}
Les langages pouvant être utilisé sont les suivants : 
\begin{itemize}
	\item C++
	\item PHP
	\item Python
	\item Ruby
	\item HTML/CSS
	\item java
	\item javascript
\end{itemize}

\section{Technologie retenue}
La technologie retenue est : site web classique + base de données avec comme langage HTML/CSS, php, javascript et phpmyadmin pour la base de données.
Le framework retenue est : ASP.NET MVC
\textbf{information supplémentaire}
Ce logiciel sera compatible sur firefox et chrome. Les compatibilités IE et Edge ne seront pas prise en compte dans le développement.
Les normes web W3c seront respecté.

\section{Organisation du projet}
Ce projet sera découpé en plusieurs jalon.
Tout les jalons seront définis plus tard ...
\begin{itemize}
	\item Jalon 1 : Création de la base de données
	\item Jalon 2 : Création de la V1 du site
	\item Jalon 3 : Création de la V1 du script PHP
	\item Jalon 4 : Connexion entre le script et la base de données
	\item Jalon 5 : Amélioration des pages web (insertion des différentes fonctionnalités) + ajout du CSS
	\item Jalon 6 : Création d'un script de remplissage de la base
	\item Jalon 7 : ?
	\item Jalon Bonus : Amélioration du CSS + ajout du script d'animation JavaScript
	\item Jalon extra bonus : Ajout compatibilité smartphone Android
\end{itemize}

\section{Conclusion}
Durée estimé du projet totale : 2-3 mois

\end{document}